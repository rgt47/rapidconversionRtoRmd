% Options for packages loaded elsewhere
\PassOptionsToPackage{unicode}{hyperref}
\PassOptionsToPackage{hyphens}{url}
\PassOptionsToPackage{dvipsnames,svgnames,x11names}{xcolor}
%
\documentclass[
  letterpaper,
  DIV=11,
  numbers=noendperiod]{scrartcl}

\usepackage{amsmath,amssymb}
\usepackage{iftex}
\ifPDFTeX
  \usepackage[T1]{fontenc}
  \usepackage[utf8]{inputenc}
  \usepackage{textcomp} % provide euro and other symbols
\else % if luatex or xetex
  \usepackage{unicode-math}
  \defaultfontfeatures{Scale=MatchLowercase}
  \defaultfontfeatures[\rmfamily]{Ligatures=TeX,Scale=1}
\fi
\usepackage{lmodern}
\ifPDFTeX\else  
    % xetex/luatex font selection
\fi
% Use upquote if available, for straight quotes in verbatim environments
\IfFileExists{upquote.sty}{\usepackage{upquote}}{}
\IfFileExists{microtype.sty}{% use microtype if available
  \usepackage[]{microtype}
  \UseMicrotypeSet[protrusion]{basicmath} % disable protrusion for tt fonts
}{}
\makeatletter
\@ifundefined{KOMAClassName}{% if non-KOMA class
  \IfFileExists{parskip.sty}{%
    \usepackage{parskip}
  }{% else
    \setlength{\parindent}{0pt}
    \setlength{\parskip}{6pt plus 2pt minus 1pt}}
}{% if KOMA class
  \KOMAoptions{parskip=half}}
\makeatother
\usepackage{xcolor}
\setlength{\emergencystretch}{3em} % prevent overfull lines
\setcounter{secnumdepth}{5}
% Make \paragraph and \subparagraph free-standing
\makeatletter
\ifx\paragraph\undefined\else
  \let\oldparagraph\paragraph
  \renewcommand{\paragraph}{
    \@ifstar
      \xxxParagraphStar
      \xxxParagraphNoStar
  }
  \newcommand{\xxxParagraphStar}[1]{\oldparagraph*{#1}\mbox{}}
  \newcommand{\xxxParagraphNoStar}[1]{\oldparagraph{#1}\mbox{}}
\fi
\ifx\subparagraph\undefined\else
  \let\oldsubparagraph\subparagraph
  \renewcommand{\subparagraph}{
    \@ifstar
      \xxxSubParagraphStar
      \xxxSubParagraphNoStar
  }
  \newcommand{\xxxSubParagraphStar}[1]{\oldsubparagraph*{#1}\mbox{}}
  \newcommand{\xxxSubParagraphNoStar}[1]{\oldsubparagraph{#1}\mbox{}}
\fi
\makeatother

\usepackage{color}
\usepackage{fancyvrb}
\newcommand{\VerbBar}{|}
\newcommand{\VERB}{\Verb[commandchars=\\\{\}]}
\DefineVerbatimEnvironment{Highlighting}{Verbatim}{commandchars=\\\{\}}
% Add ',fontsize=\small' for more characters per line
\usepackage{framed}
\definecolor{shadecolor}{RGB}{241,243,245}
\newenvironment{Shaded}{\begin{snugshade}}{\end{snugshade}}
\newcommand{\AlertTok}[1]{\textcolor[rgb]{0.68,0.00,0.00}{#1}}
\newcommand{\AnnotationTok}[1]{\textcolor[rgb]{0.37,0.37,0.37}{#1}}
\newcommand{\AttributeTok}[1]{\textcolor[rgb]{0.40,0.45,0.13}{#1}}
\newcommand{\BaseNTok}[1]{\textcolor[rgb]{0.68,0.00,0.00}{#1}}
\newcommand{\BuiltInTok}[1]{\textcolor[rgb]{0.00,0.23,0.31}{#1}}
\newcommand{\CharTok}[1]{\textcolor[rgb]{0.13,0.47,0.30}{#1}}
\newcommand{\CommentTok}[1]{\textcolor[rgb]{0.37,0.37,0.37}{#1}}
\newcommand{\CommentVarTok}[1]{\textcolor[rgb]{0.37,0.37,0.37}{\textit{#1}}}
\newcommand{\ConstantTok}[1]{\textcolor[rgb]{0.56,0.35,0.01}{#1}}
\newcommand{\ControlFlowTok}[1]{\textcolor[rgb]{0.00,0.23,0.31}{\textbf{#1}}}
\newcommand{\DataTypeTok}[1]{\textcolor[rgb]{0.68,0.00,0.00}{#1}}
\newcommand{\DecValTok}[1]{\textcolor[rgb]{0.68,0.00,0.00}{#1}}
\newcommand{\DocumentationTok}[1]{\textcolor[rgb]{0.37,0.37,0.37}{\textit{#1}}}
\newcommand{\ErrorTok}[1]{\textcolor[rgb]{0.68,0.00,0.00}{#1}}
\newcommand{\ExtensionTok}[1]{\textcolor[rgb]{0.00,0.23,0.31}{#1}}
\newcommand{\FloatTok}[1]{\textcolor[rgb]{0.68,0.00,0.00}{#1}}
\newcommand{\FunctionTok}[1]{\textcolor[rgb]{0.28,0.35,0.67}{#1}}
\newcommand{\ImportTok}[1]{\textcolor[rgb]{0.00,0.46,0.62}{#1}}
\newcommand{\InformationTok}[1]{\textcolor[rgb]{0.37,0.37,0.37}{#1}}
\newcommand{\KeywordTok}[1]{\textcolor[rgb]{0.00,0.23,0.31}{\textbf{#1}}}
\newcommand{\NormalTok}[1]{\textcolor[rgb]{0.00,0.23,0.31}{#1}}
\newcommand{\OperatorTok}[1]{\textcolor[rgb]{0.37,0.37,0.37}{#1}}
\newcommand{\OtherTok}[1]{\textcolor[rgb]{0.00,0.23,0.31}{#1}}
\newcommand{\PreprocessorTok}[1]{\textcolor[rgb]{0.68,0.00,0.00}{#1}}
\newcommand{\RegionMarkerTok}[1]{\textcolor[rgb]{0.00,0.23,0.31}{#1}}
\newcommand{\SpecialCharTok}[1]{\textcolor[rgb]{0.37,0.37,0.37}{#1}}
\newcommand{\SpecialStringTok}[1]{\textcolor[rgb]{0.13,0.47,0.30}{#1}}
\newcommand{\StringTok}[1]{\textcolor[rgb]{0.13,0.47,0.30}{#1}}
\newcommand{\VariableTok}[1]{\textcolor[rgb]{0.07,0.07,0.07}{#1}}
\newcommand{\VerbatimStringTok}[1]{\textcolor[rgb]{0.13,0.47,0.30}{#1}}
\newcommand{\WarningTok}[1]{\textcolor[rgb]{0.37,0.37,0.37}{\textit{#1}}}

\providecommand{\tightlist}{%
  \setlength{\itemsep}{0pt}\setlength{\parskip}{0pt}}\usepackage{longtable,booktabs,array}
\usepackage{calc} % for calculating minipage widths
% Correct order of tables after \paragraph or \subparagraph
\usepackage{etoolbox}
\makeatletter
\patchcmd\longtable{\par}{\if@noskipsec\mbox{}\fi\par}{}{}
\makeatother
% Allow footnotes in longtable head/foot
\IfFileExists{footnotehyper.sty}{\usepackage{footnotehyper}}{\usepackage{footnote}}
\makesavenoteenv{longtable}
\usepackage{graphicx}
\makeatletter
\def\maxwidth{\ifdim\Gin@nat@width>\linewidth\linewidth\else\Gin@nat@width\fi}
\def\maxheight{\ifdim\Gin@nat@height>\textheight\textheight\else\Gin@nat@height\fi}
\makeatother
% Scale images if necessary, so that they will not overflow the page
% margins by default, and it is still possible to overwrite the defaults
% using explicit options in \includegraphics[width, height, ...]{}
\setkeys{Gin}{width=\maxwidth,height=\maxheight,keepaspectratio}
% Set default figure placement to htbp
\makeatletter
\def\fps@figure{htbp}
\makeatother

\KOMAoption{captions}{tableheading}
\makeatletter
\@ifpackageloaded{caption}{}{\usepackage{caption}}
\AtBeginDocument{%
\ifdefined\contentsname
  \renewcommand*\contentsname{Table of contents}
\else
  \newcommand\contentsname{Table of contents}
\fi
\ifdefined\listfigurename
  \renewcommand*\listfigurename{List of Figures}
\else
  \newcommand\listfigurename{List of Figures}
\fi
\ifdefined\listtablename
  \renewcommand*\listtablename{List of Tables}
\else
  \newcommand\listtablename{List of Tables}
\fi
\ifdefined\figurename
  \renewcommand*\figurename{Figure}
\else
  \newcommand\figurename{Figure}
\fi
\ifdefined\tablename
  \renewcommand*\tablename{Table}
\else
  \newcommand\tablename{Table}
\fi
}
\@ifpackageloaded{float}{}{\usepackage{float}}
\floatstyle{ruled}
\@ifundefined{c@chapter}{\newfloat{codelisting}{h}{lop}}{\newfloat{codelisting}{h}{lop}[chapter]}
\floatname{codelisting}{Listing}
\newcommand*\listoflistings{\listof{codelisting}{List of Listings}}
\makeatother
\makeatletter
\makeatother
\makeatletter
\@ifpackageloaded{caption}{}{\usepackage{caption}}
\@ifpackageloaded{subcaption}{}{\usepackage{subcaption}}
\makeatother

\ifLuaTeX
  \usepackage{selnolig}  % disable illegal ligatures
\fi
\usepackage{bookmark}

\IfFileExists{xurl.sty}{\usepackage{xurl}}{} % add URL line breaks if available
\urlstyle{same} % disable monospaced font for URLs
\hypersetup{
  pdftitle={Rapid conversion of draft R scripts to formal Rmd reports},
  pdfauthor={R.G. Thomas},
  colorlinks=true,
  linkcolor={blue},
  filecolor={Maroon},
  citecolor={Blue},
  urlcolor={Blue},
  pdfcreator={LaTeX via pandoc}}


\title{Rapid conversion of draft R scripts to formal Rmd reports}
\usepackage{etoolbox}
\makeatletter
\providecommand{\subtitle}[1]{% add subtitle to \maketitle
  \apptocmd{\@title}{\par {\large #1 \par}}{}{}
}
\makeatother
\subtitle{A workflow to generate presentation quality wrapper for your
working R files}
\author{R.G. Thomas}
\date{2025-07-04}

\begin{document}
\maketitle

\renewcommand*\contentsname{Table of contents}
{
\hypersetup{linkcolor=}
\setcounter{tocdepth}{3}
\tableofcontents
}

\begin{figure}[H]

{\centering \includegraphics{../../images/posts/ucsd-geisel-library.jpg}

}

\caption{Caption for your hero image - either conceptual or a preview of
main results}

\end{figure}%

\section{Introduction}\label{introduction}

In this post, we'll explore {[}topic/technique/problem{]}. This is
particularly relevant for {[}target audience{]} because
{[}motivation/problem statement{]}.

By the end of this post, you'll be able to:

\begin{itemize}
\tightlist
\item
  {[}Learning objective 1{]}
\item
  {[}Learning objective 2{]}
\item
  {[}Learning objective 3{]}
\end{itemize}

\section{Prerequisites and Setup}\label{prerequisites-and-setup}

Before we begin, ensure you have the following:

\textbf{Required Packages:}

\begin{Shaded}
\begin{Highlighting}[]
\CommentTok{\# Install required packages if not already installed}
\FunctionTok{install.packages}\NormalTok{(}\FunctionTok{c}\NormalTok{(}\StringTok{"package1"}\NormalTok{, }\StringTok{"package2"}\NormalTok{, }\StringTok{"package3"}\NormalTok{))}
\end{Highlighting}
\end{Shaded}

\textbf{Load Libraries:}

\begin{Shaded}
\begin{Highlighting}[]
\CommentTok{\# Replace with your actual packages}
\CommentTok{\# library(dplyr)}
\CommentTok{\# library(ggplot2)}
\CommentTok{\# library(readr)}
\end{Highlighting}
\end{Shaded}

\textbf{Sample Data:}

\begin{Shaded}
\begin{Highlighting}[]
\CommentTok{\# Replace with your actual data loading}
\CommentTok{\# data \textless{}{-} read\_csv("your\_data.csv")}
\CommentTok{\# data \textless{}{-} mtcars  \# Example with built{-}in data}
\end{Highlighting}
\end{Shaded}

\section{Main Section 1: {[}Descriptive
Heading{]}}\label{main-section-1-descriptive-heading}

{[}Explanation of first main concept{]}

\begin{Shaded}
\begin{Highlighting}[]
\CommentTok{\# Replace with your actual example code}
\CommentTok{\# result \textless{}{-} your\_function(data)}
\CommentTok{\# print(result)}
\end{Highlighting}
\end{Shaded}

\subsection{Subsection 1.1: {[}More Specific
Topic{]}}\label{subsection-1.1-more-specific-topic}

{[}More detailed explanation or variation{]}

\begin{figure}[H]

{\centering \includegraphics{../../images/posts/ucsd-geisel-library.jpg}

}

\caption{Optional supporting visualization}

\end{figure}%

\section{Main Section 2:
{[}Implementation/Analysis{]}}\label{main-section-2-implementationanalysis}

{[}Detailed implementation or analysis{]}

\begin{Shaded}
\begin{Highlighting}[]
\CommentTok{\# Replace with your actual advanced example}
\CommentTok{\# advanced\_result \textless{}{-} complex\_analysis(data)}
\CommentTok{\# summary(advanced\_result)}
\end{Highlighting}
\end{Shaded}

\subsection{Subsection 2.1: {[}Handling Edge
Cases{]}}\label{subsection-2.1-handling-edge-cases}

{[}Discussion of potential issues and solutions{]}

\begin{Shaded}
\begin{Highlighting}[]
\CommentTok{\# Replace with your actual error handling code}
\CommentTok{\# tryCatch(\{}
\CommentTok{\#   risky\_operation(data)}
\CommentTok{\# \}, error = function(e) \{}
\CommentTok{\#   message("Error handled: ", e$message)}
\CommentTok{\# \})}
\end{Highlighting}
\end{Shaded}

\section{Main Section 3: {[}Results/Advanced
Applications{]}}\label{main-section-3-resultsadvanced-applications}

{[}Analysis of results or advanced applications{]}

\begin{Shaded}
\begin{Highlighting}[]
\CommentTok{\# Replace with your actual final analysis}
\CommentTok{\# final\_plot \textless{}{-} ggplot(data, aes(x, y)) + }
\CommentTok{\#   geom\_point() +}
\CommentTok{\#   theme\_minimal()}
\CommentTok{\# print(final\_plot)}
\end{Highlighting}
\end{Shaded}

\begin{figure}[H]

{\centering \includegraphics{main-result-plot.png}

}

\caption{Main technical visualization showing key results}

\end{figure}%

\section{Main Section 4:
{[}Performance/Comparison{]}}\label{main-section-4-performancecomparison}

{[}Performance analysis or comparison with alternative approaches{]}

\begin{Shaded}
\begin{Highlighting}[]
\CommentTok{\# Replace with your actual benchmarking code}
\CommentTok{\# system.time(method1(data))}
\CommentTok{\# system.time(method2(data))}
\end{Highlighting}
\end{Shaded}

\section{Results and Key Findings}\label{results-and-key-findings}

Our analysis revealed several key findings:

\begin{enumerate}
\def\labelenumi{\arabic{enumi}.}
\tightlist
\item
  \textbf{{[}Key finding 1{]}}: {[}Brief explanation with numbers if
  applicable{]}
\item
  \textbf{{[}Key finding 2{]}}: {[}Brief explanation{]}
\item
  \textbf{{[}Key finding 3{]}}: {[}Brief explanation{]}
\end{enumerate}

\begin{figure}[H]

{\centering \includegraphics{summary-plot.png}

}

\caption{Summary visualization highlighting main results}

\end{figure}%

\section{Limitations and
Considerations}\label{limitations-and-considerations}

While this approach is effective, there are some considerations:

\begin{itemize}
\tightlist
\item
  \textbf{{[}Limitation 1{]}}: {[}Explanation and potential
  workarounds{]}
\item
  \textbf{{[}Limitation 2{]}}: {[}Explanation{]}
\item
  \textbf{{[}Performance considerations{]}}: {[}When this approach works
  best{]}
\end{itemize}

\section{Future Extensions}\label{future-extensions}

This work could be extended in several directions:

\begin{itemize}
\tightlist
\item
  {[}Extension idea 1{]}
\item
  {[}Extension idea 2{]}
\item
  {[}Extension idea 3{]}
\end{itemize}

\section{Conclusion}\label{conclusion}

In this post, we've demonstrated {[}brief summary of what was
accomplished{]}. The key advantages of this approach are {[}main
benefits{]}.

\textbf{Next Steps:} - Try this technique with your own data -
Experiment with different parameters - Explore the additional resources
below

I encourage you to adapt this approach to your specific use case and
share your experiences in the comments below.

\section{Additional Resources}\label{additional-resources}

\textbf{Documentation and Tutorials:} -
\href{https://example.com}{Package documentation} -
\href{https://example.com}{Related tutorial} -
\href{https://example.com}{Official vignette}

\textbf{Academic References:} - Author, A. (Year). ``Paper Title''.
\emph{Journal Name}, Volume(Issue), pages. - Author, B. (Year).
``Another Paper''. \emph{Conference Proceedings}.

\textbf{Community Resources:} - \href{https://stackoverflow.com}{Stack
Overflow discussion} - \href{https://github.com}{GitHub repository} -
\href{https://r-bloggers.com}{R-bloggers related post}

\section{Reproducibility Information}\label{reproducibility-information}

\begin{verbatim}
R version 4.5.0 (2025-04-11)
Platform: aarch64-apple-darwin20
Running under: macOS Sequoia 15.5

Matrix products: default
BLAS:   /Library/Frameworks/R.framework/Versions/4.5-arm64/Resources/lib/libRblas.0.dylib 
LAPACK: /Library/Frameworks/R.framework/Versions/4.5-arm64/Resources/lib/libRlapack.dylib;  LAPACK version 3.12.1

locale:
[1] en_US.UTF-8/en_US.UTF-8/en_US.UTF-8/C/en_US.UTF-8/en_US.UTF-8

time zone: America/Los_Angeles
tzcode source: internal

attached base packages:
[1] stats     graphics  grDevices utils     datasets  methods   base     

loaded via a namespace (and not attached):
 [1] compiler_4.5.0    fastmap_1.2.0     cli_3.6.5         tools_4.5.0      
 [5] htmltools_0.5.8.1 yaml_2.3.10       rmarkdown_2.29    knitr_1.50       
 [9] jsonlite_2.0.0    xfun_0.52         digest_0.6.37     rlang_1.1.6      
[13] evaluate_1.0.3   
\end{verbatim}

\section{Appendix: {[}Optional Detailed
Information{]}}\label{appendix-optional-detailed-information}

\subsection{Appendix A: Complete Code}\label{appendix-a-complete-code}

\begin{Shaded}
\begin{Highlighting}[]
\CommentTok{\# Complete code for easy reproduction {-} replace with your actual code}
\CommentTok{\# library(your\_packages)}
\CommentTok{\# data \textless{}{-} load\_your\_data()}
\CommentTok{\# results \textless{}{-} your\_analysis(data)}
\CommentTok{\# plot(results)}
\end{Highlighting}
\end{Shaded}

\subsection{Appendix B: Mathematical
Details}\label{appendix-b-mathematical-details}

{[}Detailed mathematical explanations or derivations{]}

\subsection{Appendix C: Additional
Data}\label{appendix-c-additional-data}

{[}Additional tables, charts, or data summaries{]}

\begin{center}\rule{0.5\linewidth}{0.5pt}\end{center}

\emph{Have questions or suggestions? Feel free to reach out on
\href{https://twitter.com/yourhandle}{Twitter} or
\href{https://linkedin.com/in/yourprofile}{LinkedIn}. You can also find
the complete code for this analysis on
\href{https://github.com/yourusername/repository}{GitHub}.}

\textbf{About the Author:} {[}Your name{]} is a {[}your role{]}
specializing in {[}your expertise{]}. {[}Brief background and
interests.{]}




\end{document}
